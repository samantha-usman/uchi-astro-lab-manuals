\chapter{Inventing Color}

Everything glows (gives off electromagnetic radiation) when it has a temperature --- particles are wiggling around randomly (faster if it's hotter) and giving off energy as they wiggle. This \textbf{thermal radiation}, also called blackbody radiation, is the same for any object at the same temperature. This means we can take something's temperature by looking at the spectrum of light it gives off. This is great for learning about something's temperature at a distance, like stars.

Incandescent lightbulbs are so warm that they glow in the visible spectrum (like stars). In this lab, you'll investigate the radiation that is given off by a lightbulb at various temperatures, then use filters to mimic the situation astronomers are in when they observe astronomical objects with different color filters. You'll invent a metric to numerically state the temperature of the lightbulb, even without knowing the actual temperature. Finally, you will make a color image by combining colors from different wavelength
ranges!  This is the same method as is used to make beautiful space photos that
we see from NASA and other academic places.

\subsection{Team roles}

\begin{steps}
	\item \textbf{Decide on roles} for each group member.
\end{steps}

The available roles are:
\begin{itemize}
	\item Facilitator: ensures time and group focus are efficiently used
	\item Scribe: ensures work is recorded
	\item Technician: oversees apparatus assembly, usage
	\item Skeptic: ensures group is questioning itself
\end{itemize}

These roles can rotate each lab, and you will report at the end of the lab report on how it went for each role. Some members will be holding more than one role. For example, you could have the skeptic double with another role. Consider taking on a role you are less comfortable with, to gain experience and more comfort in that role.

Additionally, if you are finding the lab roles more restrictive than helpful, you can decide to co-hold some or all roles, or think of them more like functions that every team needs to carry out, and then reflect on how the team executed each function.

\subsection{Add members to Canvas lab report assignment group}

\begin{steps}
	\item On Canvas, navigate to the People section, then to the ``Groups'' tab. Scroll to a group called ``L3 Color [number]'' that isn't used and have each person in your group add themselves to that same lab group.
\end{steps}

\section{Lightbulbs}

Light can be made of many different wavelengths at the same time. It combines to form a color that we perceive. To see what colors make up light, we can split them up with a triangular prism, or with a \textit{diffraction grating}.

\begin{steps}
	\item Use the diffraction grating to view the lightbulb while the lightbulb is at full voltage (120 V). Hold up the grating in front of your eye, with the text on the frame upright. Start by looking at the lightbulb through it, then turn your feet to the left or right about 30 degrees, letting your whole body-arm-grating system rotate with your feet. You should see a rainbow. This is all the colors that make up the light coming from the lightbulb, split into different wavelengths.
\end{steps}

\subsection{Orientation to the digital spectrometer}

\begin{framed}
	\textbf{Warning! Fragile Equipment!} If the fiber optic cable is bent into a circle of less than 9 cm (6 inch) diameter, then the fiber inside may break.
	
	Also, the blue end cap should be replaced on the end of the fiber optic cable when you are done using it, to protect from dust and debris entering.
\end{framed}

A digital spectrometer also uses a diffraction grating, but instead of collecting the dispersed light on a screen to be viewed by people, it collects the light with a charge-coupled device (CCD), an array of light-sensitive pixels much like a digital camera. It then translates the position on the CCD to individual wavelengths and displays a plot of intensity vs. wavelength on a computer. %TODO (see Figure~???)
Another difference is that we use an optical fiber to collect the light.

Here are a few guidelines:
\begin{itemize}
	
	\item \textbf{Background subtraction.} When you are observing the spectrum of something, there may be stray light entering that is not coming from the thing you are studying. To remove this \textit{background} from the plot, select the gray lightbulb icon in the toolbar. This records the current plot as the background. To display the plot while subtracting the background, select the icon with the minus-lightbulb. To go back to viewing the full spectrum including the background, select the blue S icon.

	\item \textbf{Oversaturation}. If, when you zoom out all the way, the plot includes a flat line near the top of the plot, this means that the pixels of the image sensor are recording their maximum value, and they cannot tell you the actual intensity. Reduce the intensity by either reducing the integration time (upper-left corner) or moving the fiber optic cable off-center, so that it receives less light. Note that if you change either of these, then the background frame is no longer correct and must be remeasured.

	\item \textbf{Saving data.}
	
	\begin{itemize}
		\item Save the images of spectra and numeric files that you generated with SpectraSuite
	during your experiments on a USB stick, so that you can use them at home during
	preparation of your report (you can also email them as attachments from your
	computer at the end of the lab).
	
	\item You should open a word processor document and a spreadsheet document in which you can save your measured
	spectra at the beginning of your work. To save an image of your graph, click on
	the fourth icon from the left in the Spectrum IO controls. %TODO (Fig.~???)
	This will copy
	an image of the graph to the clipboard. Then in your word processor, paste the image by pressing
	Ctrl-V.
	
	\item To save spectrum in the digital form, click on the third from the left icon in
	Spectrum IO controls (to the right of print icon). This copies it to clipboard. In
	Excel file make sure you are in a new sheet and press Ctrl-V. This should create\item How does the shape change when the brightness increases? \textbf{Record your answer.}
	to columns of numbers: wavelength (in nm) and counts for your spectrum.
	
	\item An alternative way to save the data is to click on the floppy disk icon in the
	Spectrum IO controls. The format must be ``Tab delimiter, no header''. The
	writing directory must be specified (``Browse'' button). The spectrum is saved in a
	text file (.txt) as two columns, the first column giving the wavelength in nm, the
	second column the corresponding intensity. This file can be imported into a spreadsheet or plotting program.
	\end{itemize}
\end{itemize}

\subsection{Observing the lightbulb - see the rainbow!}\label{ic:sec:exploring}

\begin{steps}
	
	\item Using the spectrometer, observe the spectrum created by a lightbulb at various brightnesses / voltages.  What's the overall shape? \textbf{Record your answer.}
	
	\item How does the shape change when the brightness increases? How about the total radiated power (area under the curve?) \textbf{Record your answer.}
	
	\item How does the peak frequency change? \textbf{Record your answer.}
	
	\item How does the visual color of the bulb change as it gets brighter? \textbf{Record your answer.}

	\item \item\label{ic:step:load-sim} Load the interactive thermal radiation simulation
	at
	\url{https://phet.colorado.edu/sims/html/blackbody-spectrum/latest/blackbody-spectrum_en.html}
	
	\item Play with the controls and see what happens. Discuss among your team.
	
	\item\label{ic:step:pattern} Based on your observations of the lightbulb spectrum and of the sim, report the pattern you see: what happens when the temperature increases? \textbf{Record your answer.} You will test the consistency of this pattern later in the lab.
	
	\item At what peak wavelength do you radiate? Use the sim to determine this and 
	estimate your uncertainty. \textbf{Record your findings.}
	
	\item Humans can see in the range of 380 to 700 nanometers (aka $10^{-9}$ m). Does your answer for the peak wavelength from the previous step make sense? For example, if you are in a dark room, would you be able to spot the human from their glow? \textbf{Record your answer.}

\end{steps}

\section{Make the rainbow! (But please don't taste the rainbow!)}

Cameras have come a long way since the days of developing film, but sensors are not as smart you might think. They register light, but they can't usually tell the color (i.e. wavelength) of the incoming photons. So then how do you get a color image? In astronomy (and in your cell phone!) color images are generated by measuring the amount of light at specific colors and then combining these measurements to create a colorful image. A filter is used to select bands of color to allow through. In consumer cameras and phone cameras, these filters are permanently attached to the front of individual pixels. In astronomical imaging, there is no permanent filter, and different filters are moved into place.

\begin{steps}
	\item View the lightbulb's spectrum with the diffraction grating again, this time alternately holding the red and green filters between the grating and the bulb. Observe and \textbf{record} the effect of the filter on the spectrum.
	
	\item Use the red and green filters in front of the fiber optic input to the spectrometer to see how they affect the spectrum that is received. \textbf{Save a graph of the spectrum with each filter and include in your report.}
\end{steps}

In a regular astronomical image, each pixel gives just one value --- the number of counts detected in that pixel, regardless of the wavelength of the photon detected. We can treat the fiber optic as a single-pixel camera if we count up the total number of counts detected. To find that total number, you can find the area under the curve in the plot.

\begin{steps}
	\item With no filter, add up the total number of counts detected by calculating the area under the curve. To do this in an approximate way, count the number of boxes underneath the curve, then multiply this by the height of one box (in counts per nanometer) and by the width of one box (in nanometers). \textbf{Record this value.} (to do this more precisely, you can press the copy icon to paste the data into a spreadsheet)
	
	\item Without moving the fiber optic or lightbulb, do the same for the spectrum using the red and green filters separately. Note that the values are different for different filters. This difference (for example, the green value divided by the red value) can tell us a quantitative number for what color this object is. \textbf{Record the red, green, and $g/r$ values.}
\end{steps}

Now you have the value of our single-pixel camera for the case of clear (no filter), red, and green filters. Time to revisit the pattern found above in Step \ref{ic:step:pattern}.

\begin{steps}
	\item If that pattern from Step \ref{ic:step:pattern} is true, what should happen to the relative values of red and green as the voltage is increased? Should one increase more than the other? What should happen to the quantitative color $g/r$? \textbf{Record your answers.}
	
	\item Perform an experiment to test whether this prediction is supported. \textbf{Record your procedure, analysis, and results.}
	
	Hints:
	
	\begin{enumerate}
		\item Choose 5 or so different voltages you'd like to use for the adjustable lamp.
		Set your lamp to the \emph{highest} of these 5 voltages. Move the cable as close to the lightbulb as possible
		\emph{without} saturating the spectrometer, i.e. without the spectrum "leveling
		out" in the spectrometer program. Mark the position of the clamp and \textbf{do
			not move} the cable from now on.
		
		\item For each voltage, measure the total counts for the red and green filters, then divide them to get your $g/r$ value for each voltage.
		
		\item Plot these and judge whether your prediction above was suppported.
	\end{enumerate}
\end{steps}

In next week's lab, you will use this quantitative color index to determine the age of a star cluster.

\section{Creating a color image}

To create the beautiful color astronomy images we like to see, one must manually combine images taken with different filters. There are always choices to be made that will change how the combined image looks.

\begin{steps}
	\item\label{ic:step:color-image} Using the directions below, create a
	false color image of either the Orion nebula (M42), a stellar nursery, or the
	Crab nebula (M1), the remnant of a supernova. You can find the relevant images
	on Canvas in the Lab module.  \end{steps}

Now you'll create a color image from three separate images of the same target with r (red), g (green), and i (infrared) filters. Since one of the filters is infrared, and we only have red, green, and blue colors available in DS9, then you will assign the image taken in the infrared to the color blue. This makes this image a \textit{false color image}, since the blue we will see in it does not correspond to the actual wavelength of the light captured.

\begin{framed}
	\emph{Fun fact:}
	This is actually how photos from telescopes like Hubble are taken! Astronomers
	take photos of celestial bodies in multiple filter ranges, than artificially
	color them to make beautiful space images!
\end{framed}


\textbf{Loading and manipulating images in ds9 consists of:}
\begin{itemize}
\item loading an image  (file $>$ open)
\item setting lower, upper limits (z1,z2) on an image  (scale $>$ various algorithms; use scale $>$ scale parameters for full control). See Figures \ref{ic:fig:z-min-max}--\ref{ic:fig:z-small} for examples.
\item controlling the intensity mapping within those bounds (mouse right click-and-hold and drag)
\end{itemize}

\begin{figure}
	\includegraphics[width=0.5\textwidth]{inventing-color/z-min-max}
	\includegraphics[width=0.5\textwidth]{inventing-color/z-mid}
	\caption{Proper choice of data ranges is important. The default for ds9 is often the min/max values in the image, which can be a poor choice if there are outlier pixels, as shown here on the left. The red line shows the lower limit z1, which is mapped to no color (black here), and the green line is the upper limit z2, mapped to full color (white here). Pixel values between these are shown in various brightnesses of the color. On the right, z1 and z2 are more tuned to the distribution of pixel values, which more effectively uses the dynamic range of the display for pixel values where there are significant amounts of data.}\label{ic:fig:z-min-max}
\end{figure}

\begin{figure}
		\includegraphics[width=0.5\textwidth]{inventing-color/z-small}
		\caption{Smaller values of z2 will emphasize fainter values in the target. Compare the image to the left to the right-hand image above.}\label{ic:fig:z-small}
\end{figure}

\textbf{You can change the zoom and center location in an image by} 
\begin{itemize}
\item moving around in image (mouse middle click if edit$>$point is set[the default], or edit$>$pan and mouse left click)
\item zooming in and out (mouse wheel, zoom$>$ +,- etc.)
\end{itemize}

\textbf{To build a color image}
\begin{itemize}
	\item Identify and download from Canvas your three different filter image FITS files.
\item Open a color (rather than monochrome) frame:  Frame $>$ new rgb
\item Open the red, green, infrared files using the rgb subwindow to select which channel you are working in, and then scale and control intensities on each one. 
\item There are many possible ways to scale the images. Some testing suggests that choosing Scale $>$ ASINH (or Linear or Square Root as other choices) and Scale $>$ 99.5\% (or maybe 99\% or 98\%)  produces reasonable results. Experiment!
\item One thing to note: the rgb subwindow allows you to control how the images are aligned spatially via the “align” menu at the top. There are three relevant choices: “WCS”, “Image” or “Physical”.  The latter two should give the same result in this instance. “WCS” alignment uses information in the image header that has been added by the processing pipeline, that establishes a World Coordinate System (this tells ds9 and other programs how pixel x,y values are mapped into the sky coordinates - typically Right Ascension [east/west] and Declination [north/south]). The default is WCS and should work fine, if the images processed correctly. If that doesn’t look good, you could try the others. If that still doesn’t look good, note that you tried your best, and give an example of how it didn’t work will in either mode. Examples of good and bad image alignment are shown in Figure \ref{ic:fig:rgb-bad}.
\end{itemize}

\begin{figure}
	\includegraphics[width=0.5\textwidth]{inventing-color/rgb-bad}
	\includegraphics[width=0.5\textwidth]{inventing-color/rgb-good}
	\caption{Zoom-in on a color image, showing poor (left) and good (right) image alignment across filters. Note how objects are shifted between different color channels in the poorly aligned image.}\label{ic:fig:rgb-bad}
\end{figure}

\section{Report checklist and grading}

Each item below is worth 10 points. See Appendix\ \ref{cha:lab-report-format} for guidance on writing the report and formatting tables and graphs.

\begin{enumerate}
	\item Qualitative observations with the digital spectrometer (Steps 2--5).
	\item The pattern / relationship between temperature and spectrum (Step \ref{ic:step:pattern}).
	\item Effect of filters on spectrum (Steps 7--8).
	\item Total intensity calculations and plots of spectra with clear, red, and green filters. (Steps 9--10).
	\item Prediction of what should happen to $g/r$ with temperature increase, and procedure, analysis, results of testing that prediction (Steps 11--12).
	\item Your beautiful color image (Section 3.2).
	\item Discuss the findings and reflect deeply on the quality and importance of
	the findings. This can be both in the frame of a scientist conducting the
	experiment (“What did the experiment tell us about the world?”) and in the
	frame of a student (“What skills or mindsets did I learn?”).
	\item Write a 100--200 word paragraph reporting back from each of the four roles: facilitator, scribe, technician, skeptic. Where did you see each function happening during this lab, and where did you see gaps? What successes and challenges in group functioning did you have? What do you want to do differently next time?
\end{enumerate}
